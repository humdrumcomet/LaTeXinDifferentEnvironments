\usepackage[mode=buildnew, subpreambles=false]{standalone}
\usepackage{import} % needed for importing with path consideration
\usepackage{geometry} % useful for defining page geometries
\usepackage{hyperref} % used for creating hyperlinks in documents. Both to the web and within the document itself
\makeatletter
\@ifclassloaded{beamer}{
  \typeout{
    -------------------------------------------------
    -------------------------------------------------
    this file is a beamer file, skipping AMS packages
    -------------------------------------------------
    -------------------------------------------------
  }
}{
  \typeout{
    ----------------------------------------------------
    ----------------------------------------------------
    this file is not a beamer file, loading AMS packages
    ----------------------------------------------------
    ----------------------------------------------------
  }
  \usepackage[tbtags]{amsmath} % for typesetting math (American Mathematical Society)
  \usepackage{amsfonts} % fonts and mathematical symbols
  \usepackage{amssymb} % more mathematical symbols
}
\makeatother
% Not needed with lualatex
% \usepackage[utf8]{inputenc} % how to treat the written file (as utf8)
% \usepackage{morewrites} % important because with the glossaries, and beamers own files it is writing out, latex can use up its own hardcoded limit
\usepackage[T1]{fontenc} % the encoding for the output file T1 is the most common, includes accents and many other commonly needed/used characters
\usepackage{hologo}
\usepackage[style=ieee,backend=biber]{biblatex} % for handling bibliographies
\usepackage{float} % added control over
\usepackage{graphicx} % tools for inclusion of graphics
\usepackage{booktabs} % adding commands to improve the look of tables
\usepackage{csvsimple} % simplify table creation by importing .csv files directly
\usepackage{siunitx} % consistent notation and correct formatting of units
\DeclareSIUnit{\torr}{Torr} % Custom unit definition
\usepackage{minted} % inclusion of code blocks with syntax highlighting and
\setminted{linenos, autogobble, fontsize=\footnotesize, breaklines=true} % set global options for minted environments
\usepackage{chemformula} % for writing chemical formulae
\usepackage[useregional]{datetime2}
\usepackage{catchfilebetweentags} % taking elements from between tags

\usepackage[debug, toc, section=section, acronym, symbols]{glossaries} % Glossaries package
\usepackage{tikz}
    \usetikzlibrary{math, arrows, circuits.ee.IEC, positioning, shapes.arrows, shapes.geometric}
\usepackage{pgfplots}
    \pgfplotsset{compat=newest, compat/show suggested version=false}
    \usepgfplotslibrary{groupplots}
\usepackage[siunitx,american voltages, american currents, RPvoltages]{circuitikz}

% A command for less cramped nested fractions
\newcommand\ddfrac[2]{\frac{\displaystyle #1}{\displaystyle #2}}

%Get rounded wire jumps
\tikzset{
    declare function={% in case of CVS which switches the arguments of atan2
        atan3(\a,\b)=ifthenelse(atan2(0,1)==90, atan2(\a,\b), atan2(\b,\a));},
        kinky cross radius/.initial=+.125cm,
        @kinky cross/.initial=+, kinky crosses/.is choice,
        kinky crosses/left/.style={@kinky cross=-},kinky crosses/right/.style={@kinky cross=+},
        kinky cross/.style args={(#1)--(#2)}{
        to path={
          let \p{@kc@}=($(\tikztotarget)-(\tikztostart)$),
              \n{@kc@}={atan3(\p{@kc@})+180} in
          -- ($(intersection of \tikztostart--{\tikztotarget} and #1--#2)!%
                 \pgfkeysvalueof{/tikz/kinky cross radius}!(\tikztostart)$)
          arc [ radius     =\pgfkeysvalueof{/tikz/kinky cross radius},
                start angle=\n{@kc@},
                delta angle=\pgfkeysvalueof{/tikz/@kinky cross}180 ]
          -- (\tikztotarget)}}
      }

% SI unit pkg conf
\sisetup{
    load-configurations = abbreviations,
    detect-family = true,
    per-mode = reciprocal
}%
\DeclareSIUnit{\torr}{Torr}

% Bibliography .bib file locations should maybe be local?
\addbibresource[location=local]{./accelerators.bib}

% For using the standalone package
\newboolean{standaloneFlag}
\setboolean{standaloneFlag}{true}
% Add Constants list using glossary
\newglossary[cgls]{constants}{cstog}{cstig}{Constants}

% Alphabetize glossary and acronyms list
\makeglossaries


% Command to conditionally typeset a bibliography.
\newcommand{\standaloneBib}{%%
  \ifthenelse{\boolean{standaloneFlag}}%%
  {\printbibliography[heading=bibintoc]
        \printglossary[type=symbols]
        \printglossary[type=acronymtype]
  \printglossary[type=main]}{}}

\DTMsavedate{presentation}{2022-11-19}
