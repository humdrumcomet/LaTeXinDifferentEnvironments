%%%%%%%%%%%%%%%%%%%%%%%%%%%%%%%%%%%%%%%%%%%%%%%%%%%%%%%%%%%%%%%%%%%%%%%%%%%%%%%%%%%%%%%%%%%%%%%%%%%%%%%%%%%%%%%%%%%%%%%%%%%%%%%%%%%%%%%%%%%%%%%%%%%%%%%%%%%%%%%%%
%%%%%%%%%% LOAD PACKAGES %%%%%%%%%%%%%%%%%%%%%%%%%%%%%%%%%%%%%%%%%%%%%%%%%%%%%%%%%%%%%%%%%%%%%%%%%%%%%%%%%%%%%%%%%%%%%%%%%%%%%%%%%%%%%%%%%%%%%%%%%%%%%%%%%%%%%%%%
%%%%%%%%%%%%%%%%%%%%%%%%%%%%%%%%%%%%%%%%%%%%%%%%%%%%%%%%%%%%%%%%%%%%%%%%%%%%%%%%%%%%%%%%%%%%%%%%%%%%%%%%%%%%%%%%%%%%%%%%%%%%%%%%%%%%%%%%%%%%%%%%%%%%%%%%%%%%%%%%%
\usepackage{geometry} % useful for defining page geometries
\usepackage{hyperref} % used for creating hyperlinks in documents. Both to the web and within the document itself
\usepackage[mode=build, subpreambles=false]{standalone} % for using the standalone package for import
\makeatletter
\@ifclassloaded{beamer}{
  \typeout{
    -------------------------------------------------
    -------------------------------------------------
    this file is a beamer file, skipping AMS packages
    -------------------------------------------------
    -------------------------------------------------
  }
}{
  \typeout{
    ----------------------------------------------------
    ----------------------------------------------------
    this file is not a beamer file, loading AMS packages
    ----------------------------------------------------
    ----------------------------------------------------
  }
  \usepackage[tbtags]{amsmath} % for typesetting math (American Mathematical Society)
  \usepackage{amsfonts} % fonts and mathematical symbols
  \usepackage{amssymb} % more mathematical symbols
}
\makeatother
% These packages aren't needed when compiling under LuaLaTeX
% \usepackage[utf8]{inputenc} % how to treat the written file (as utf8)
% \usepackage{morewrites} % important because with the glossaries, latex can use up its own hardcoded limit of files it can write out
% \usepackage[T1]{fontenc} % the encoding for the output file T1 is the most common, includes accents and many other commonly needed/used characters
%%%%%%%%%%%%%%%%%%%%%%%%%%%%%%%%%%%%%%%%%%%%%%%%%%%%%%%%%%%%%%%%%%%%%%%%%%%%%%%%%%%%%%%%%%%%%%%%%%%%%%%%%%%%%%%%%%%%%%%%%%%%%%%%%%%%%%%%%%%%%%%%%%%%%%%%%%%%%%%%%
\usepackage[style=ieee,backend=biber]{biblatex} % for handling bibliographies
\usepackage{float} % added control over
\usepackage{graphicx} % tools for inclusion of graphics
\usepackage{booktabs} % adding commands to improve the look of tables
\usepackage{csvsimple} % simplify table creation by importing .csv files directly
\usepackage{siunitx} % consistent notation and correct formatting of units
\usepackage{minted} % inclusion of code blocks with syntax highlighting and
\usepackage{xcolor} % to access the named colour LightGray
\usepackage{chemformula} % for writing chemical formulae
\usepackage[useregional]{datetime2} % facilitate date and time typesetting
\usepackage{catchfilebetweentags} % taking elements from between tags
\usepackage[debug, toc, section=section, acronym, symbols]{glossaries} % Glossaries package
\usepackage{tikz} % to produce tikz vector graphics images in your document
\usepackage{pgfplots} % to produce tikz/pgf vector graphic plots within your document
\usepackage[siunitx,american voltages, american currents, RPvoltages]{circuitikz} % circuits with tikz
\usepackage{hologo}
\usepackage{luacode}

%%%%%%%%%%%%%%%%%%%%%%%%%%%%%%%%%%%%%%%%%%%%%%%%%%%%%%%%%%%%%%%%%%%%%%%%%%%%%%%%%%%%%%%%%%%%%%%%%%%%%%%%%%%%%%%%%%%%%%%%%%%%%%%%%%%%%%%%%%%%%%%%%%%%%%%%%%%%%%%%%
%%%%%%%%%% PACKAGE SETUP %%%%%%%%%%%%%%%%%%%%%%%%%%%%%%%%%%%%%%%%%%%%%%%%%%%%%%%%%%%%%%%%%%%%%%%%%%%%%%%%%%%%%%%%%%%%%%%%%%%%%%%%%%%%%%%%%%%%%%%%%%%%%%%%%%%%%%%%
%%%%%%%%%%%%%%%%%%%%%%%%%%%%%%%%%%%%%%%%%%%%%%%%%%%%%%%%%%%%%%%%%%%%%%%%%%%%%%%%%%%%%%%%%%%%%%%%%%%%%%%%%%%%%%%%%%%%%%%%%%%%%%%%%%%%%%%%%%%%%%%%%%%%%%%%%%%%%%%%%
% minted
\definecolor{LightGray}{gray}{0.9}
\colorlet{PresentationBlack}{black!2}
\setminted{linenos=false, autogobble, fontsize=\tiny, breaklines=true, bgcolor=LightGray, breakbefore=\{} % global default options for minted
\setmintedinline{bgcolor={}, fontsize=\small} % global default options for inline minted

% tikz
\usetikzlibrary{math, arrows, circuits.ee.IEC, positioning, shapes.arrows, shapes.geometric, automata, fadings}

% pgfplots
\pgfplotsset{compat=newest, compat/show suggested version=false}
\usepgfplotslibrary{groupplots}

% siunitx
\sisetup{
detect-family = true,
detect-weight = true,
per-mode = reciprocal,
input-digits = { 0123456789\pi\dots },
input-comparators = { <=>\approx\ge\geq\gg\le\leq\ll\sim\gtrsim\lesssim },
table-auto-round,
table-align-comparator = true,
}%
\DeclareSIUnit{\torr}{Torr} % Custom unit definition

% Bibliography .bib file location
\addbibresource[location=local]{\bibPath/references.bib}

% Add Constants list using glossary
\newglossary[cgls]{constants}{cstog}{cstig}{Constants}

% Alphabetize glossary and acronyms list
\makeglossaries

%%%%%%%%%%%%%%%%%%%%%%%%%%%%%%%%%%%%%%%%%%%%%%%%%%%%%%%%%%%%%%%%%%%%%%%%%%%%%%%%%%%%%%%%%%%%%%%%%%%%%%%%%%%%%%%%%%%%%%%%%%%%%%%%%%%%%%%%%%%%%%%%%%%%%%%%%%%%%%%%%
%%%%%%%%%% ADDITIONAL SETUP %%%%%%%%%%%%%%%%%%%%%%%%%%%%%%%%%%%%%%%%%%%%%%%%%%%%%%%%%%%%%%%%%%%%%%%%%%%%%%%%%%%%%%%%%%%%%%%%%%%%%%%%%%%%%%%%%%%%%%%%%%%%%%%%%%%%%
%%%%%%%%%%%%%%%%%%%%%%%%%%%%%%%%%%%%%%%%%%%%%%%%%%%%%%%%%%%%%%%%%%%%%%%%%%%%%%%%%%%%%%%%%%%%%%%%%%%%%%%%%%%%%%%%%%%%%%%%%%%%%%%%%%%%%%%%%%%%%%%%%%%%%%%%%%%%%%%%%
% A command for less cramped nested fractions
\newcommand\ddfrac[2]{\frac{\displaystyle #1}{\displaystyle #2}}

% Front matter, main matter, and back matter definitions (not needed for document class book or memoire)
\def\frontmatter{%
 \pagenumbering{roman}
 \setcounter{page}{1}
 \renewcommand{\thesection}{\roman{section}}
}%
\def\mainmatter{%
 \pagenumbering{arabic}
 \setcounter{page}{1}
 \setcounter{section}{0}
 \renewcommand{\thesection}{\arabic{section}}
}%
\def\backmatter{%
 \setcounter{section}{0}
 \renewcommand{\thesection}{\alph{section}}
}%

% taken from tikz  manual v3.1.10 sec. 106.5.3 Command for declaring new shapes pg 1149
\makeatletter
\pgfdeclareshape{document}{
\inheritsavedanchors[from=rectangle] % this is nearly a rectangle
\inheritanchorborder[from=rectangle]
\inheritanchor[from=rectangle]{center}
\inheritanchor[from=rectangle]{north}
\inheritanchor[from=rectangle]{south}
\inheritanchor[from=rectangle]{west}
\inheritanchor[from=rectangle]{east}
% ... and possibly more
\backgroundpath{% this is new
% store lower right in xa/ya and upper right in xb/yb
\southwest \pgf@xa=\pgf@x \pgf@ya=\pgf@y
\northeast \pgf@xb=\pgf@x \pgf@yb=\pgf@y
% compute corner of ‘‘flipped page’’
\pgf@xc=\pgf@xb \advance\pgf@xc by-10pt % this should be a parameter
\pgf@yc=\pgf@yb \advance\pgf@yc by-10pt
% construct main path
\pgfpathmoveto{\pgfpoint{\pgf@xa}{\pgf@ya}}
\pgfpathlineto{\pgfpoint{\pgf@xa}{\pgf@yb}}
\pgfpathlineto{\pgfpoint{\pgf@xc}{\pgf@yb}}
\pgfpathlineto{\pgfpoint{\pgf@xb}{\pgf@yc}}
\pgfpathlineto{\pgfpoint{\pgf@xb}{\pgf@ya}}
\pgfpathclose
% add little corner
\pgfpathmoveto{\pgfpoint{\pgf@xc}{\pgf@yb}}
\pgfpathlineto{\pgfpoint{\pgf@xc}{\pgf@yc}}
\pgfpathlineto{\pgfpoint{\pgf@xb}{\pgf@yc}}
\pgfpathlineto{\pgfpoint{\pgf@xc}{\pgf@yc}}
}
}
\makeatother

% Taken from TeX SE https://tex.stackexchange.com/questions/103688/folded-paper-shape-tikz
\tikzstyle{doc}=[%
draw,
thick,
align=center,
color=black,
shape=document,
minimum width=20mm,
minimum height=28.2mm,
shape=document,
inner sep=2ex,
% text width=10mm,
]

%Get rounded wire jumps
\tikzset{
    declare function={% in case of CVS which switches the arguments of atan2
        atan3(\a,\b)=ifthenelse(atan2(0,1)==90, atan2(\a,\b), atan2(\b,\a));},
        kinky cross radius/.initial=+.125cm,
        @kinky cross/.initial=+, kinky crosses/.is choice,
        kinky crosses/left/.style={@kinky cross=-},kinky crosses/right/.style={@kinky cross=+},
        kinky cross/.style args={(#1)--(#2)}{
        to path={
          let \p{@kc@}=($(\tikztotarget)-(\tikztostart)$),
              \n{@kc@}={atan3(\p{@kc@})+180} in
          -- ($(intersection of \tikztostart--{\tikztotarget} and #1--#2)!%
                 \pgfkeysvalueof{/tikz/kinky cross radius}!(\tikztostart)$)
          arc [ radius     =\pgfkeysvalueof{/tikz/kinky cross radius},
                start angle=\n{@kc@},
                delta angle=\pgfkeysvalueof{/tikz/@kinky cross}180 ]
          -- (\tikztotarget)}},
    onslide/.code args={<#1>#2}{%
        \only<#1>{\pgfkeysalso{#2}} % \pgfkeysalso doesn't change the path
    },
    myfading/.style n args={2}{
        postaction={
            decorate,
            decoration={
                markings,
                mark=between positions 0 and \pgfdecoratedpathlength-4pt step 0.2pt with {
                    \pgfmathsetmacro\myval{
                            multiply(
                                divide(
                                    \pgfkeysvalueof{/pgf/decoration/mark info/distance from start},
                                    \pgfdecoratedpathlength
                                ),
                                100
                            )
                    };
                    \pgfsetfillcolor{#2!\myval!#1};
                    \pgfpathcircle{\pgfpointorigin}{\pgflinewidth};
                    \pgfusepath{fill};
                },
                mark=at position 1 with {\arrow[#2, >=latex, auto]{>}},
            }
        }
    },
}
\usepackage[most]{tcolorbox}
\usepackage{tikzpagenodes}

\def\myblur{4}

\makeatletter
\newtcolorbox{blur}[1][]{%
  #1,
  enhanced,
  remember,
  frame hidden,
  interior hidden,
  fonttitle=\bfseries,
  coltitle=black,
  underlay={
    \begin{tcbclipframe}
      \begin{scope}[remember picture,overlay,inner sep=0pt]
        \fill[white] (current page.south west) rectangle (current page.north east);
        \foreach \x in {-10,-7.5,...,10}{
        \foreach \y in {-10,-7.5,...,10}{
          \node[opacity=0.01] at ([yshift=\y,xshift=\x]current page.center) {\includestandalone[mode=tex, width=\textwidth]{\assetPath/Images/Tikz/texfamilytree/texfamilytree}};
        }}
      \end{scope}
    \end{tcbclipframe}
   }
}
\makeatother

\def\checkmark{\tikz\fill[scale=0.4](0,.35) -- (.25,0) -- (1,.7) -- (.25,.15) -- cycle;}

% For using the standalone package and conditionally typesetting bibliography and glossaries
\newboolean{standaloneFlag}
\setboolean{standaloneFlag}{true}
% Create command to conditionally typeset a bibliography.
\newcommand{\standaloneBib}{%%
  \ifthenelse{\boolean{standaloneFlag}}{
    \clearpage
    \printbibliography[heading=bibintoc]
    \printglossary[type=symbols]
    \printglossary[type=constants]
    \printglossary[type=acronymtype]
    \printglossary[type=main]
  }{}}

% Code for Lua search and Replace

\directlua{
    path = "\luacodePath"
    demo = require(path.."/searchandreplace")
}
\newcommand\luastringsubs[3]{%
    \directlua{
        replacementsPreTable = [[\unexpanded{#1}]]
        filePathIn = [[#2]]
        filePathOut = [[#3]]
        subbedString = stringReplaceFromFile(replacementsPreTable, filePathIn, filePathOut)
    }%
}
