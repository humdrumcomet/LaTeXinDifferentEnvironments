% arara: lualatex: { shell: true, interaction: nonstopmode }
% arara: makeglossaries
% arara: biber
% arara: lualatex: { shell: true, interaction: nonstopmode }
% arara: lualatex: { synctex: true, shell: true, interaction: nonstopmode }

\providecommand{\toplevel}{../..}
\providecommand{\importPath}{\toplevel/Shared/Imports}
\providecommand{\assetPath}{\toplevel/Assets}
\providecommand{\sharedPath}{\toplevel/Shared}

\documentclass[hidelinks, float=false, crop=false]{standalone}

\usepackage[mode=build, subpreambles=false]{standalone}
\usepackage{import} % needed for importing with path consideration
\usepackage{geometry} % useful for defining page geometries
\usepackage{hyperref} % used for creating hyperlinks in documents. Both to the web and within the document itself
\makeatletter
\@ifclassloaded{beamer}{
  \typeout{
    -------------------------------------------------
    -------------------------------------------------
    this file is a beamer file, skipping AMS packages
    -------------------------------------------------
    -------------------------------------------------
  }
}{
  \typeout{
    ----------------------------------------------------
    ----------------------------------------------------
    this file is not a beamer file, loading AMS packages
    ----------------------------------------------------
    ----------------------------------------------------
  }
  \usepackage[tbtags]{amsmath} % for typesetting math (American Mathematical Society)
  \usepackage{amsfonts} % fonts and mathematical symbols
  \usepackage{amssymb} % more mathematical symbols
}
\makeatother
% Not needed with lualatex
% \usepackage[utf8]{inputenc} % how to treat the written file (as utf8)
% \usepackage{morewrites} % important because with the glossaries, and beamers own files it is writing out, latex can use up its own hardcoded limit
\usepackage[T1]{fontenc} % the encoding for the output file T1 is the most common, includes accents and many other commonly needed/used characters
\usepackage{hologo}
\usepackage[style=ieee,backend=biber]{biblatex} % for handling bibliographies
\usepackage{float} % added control over
\usepackage{graphicx} % tools for inclusion of graphics
\usepackage{booktabs} % adding commands to improve the look of tables
\usepackage{csvsimple} % simplify table creation by importing .csv files directly
\usepackage{siunitx} % consistent notation and correct formatting of units
\DeclareSIUnit{\torr}{Torr} % Custom unit definition
\usepackage{minted} % inclusion of code blocks with syntax highlighting and
\setminted{linenos, autogobble, fontsize=\footnotesize, breaklines=true} % set global options for minted environments
\usepackage{chemformula} % for writing chemical formulae
\usepackage[useregional]{datetime2}
\usepackage{catchfilebetweentags} % taking elements from between tags

\usepackage[debug, toc, section=section, acronym, symbols]{glossaries} % Glossaries package
\usepackage{tikz}
    \usetikzlibrary{math, arrows, circuits.ee.IEC, positioning, shapes.arrows, shapes.geometric, automata}
\usepackage{pgfplots}
    \pgfplotsset{compat=newest, compat/show suggested version=false}
    \usepgfplotslibrary{groupplots}
\usepackage[siunitx,american voltages, american currents, RPvoltages]{circuitikz}

% A command for less cramped nested fractions
\newcommand\ddfrac[2]{\frac{\displaystyle #1}{\displaystyle #2}}

% Front matter, main matter, and back matter definitions
\def\frontmatter{%
 \pagenumbering{roman}
 \setcounter{page}{1}
 \renewcommand{\thesection}{\roman{section}}
}%
\def\mainmatter{%
 \pagenumbering{arabic}
 \setcounter{page}{1}
 \setcounter{section}{0}
 \renewcommand{\thesection}{\arabic{section}}
}%
\def\backmatter{%
 \setcounter{section}{0}
 \renewcommand{\thesection}{\alph{section}}
}%

%Get rounded wire jumps
\tikzset{
    declare function={% in case of CVS which switches the arguments of atan2
        atan3(\a,\b)=ifthenelse(atan2(0,1)==90, atan2(\a,\b), atan2(\b,\a));},
        kinky cross radius/.initial=+.125cm,
        @kinky cross/.initial=+, kinky crosses/.is choice,
        kinky crosses/left/.style={@kinky cross=-},kinky crosses/right/.style={@kinky cross=+},
        kinky cross/.style args={(#1)--(#2)}{
        to path={
          let \p{@kc@}=($(\tikztotarget)-(\tikztostart)$),
              \n{@kc@}={atan3(\p{@kc@})+180} in
          -- ($(intersection of \tikztostart--{\tikztotarget} and #1--#2)!%
                 \pgfkeysvalueof{/tikz/kinky cross radius}!(\tikztostart)$)
          arc [ radius     =\pgfkeysvalueof{/tikz/kinky cross radius},
                start angle=\n{@kc@},
                delta angle=\pgfkeysvalueof{/tikz/@kinky cross}180 ]
          -- (\tikztotarget)}}
      }

% SI unit pkg conf
\sisetup{
    load-configurations = abbreviations,
    detect-family = true,
    per-mode = reciprocal
}%
\DeclareSIUnit{\torr}{Torr}

% Bibliography .bib file locations should maybe be local?
\addbibresource[location=local]{./accelerators.bib}

% For using the standalone package
\newboolean{standaloneFlag}
\setboolean{standaloneFlag}{true}
% Add Constants list using glossary
\newglossary[cgls]{constants}{cstog}{cstig}{Constants}

% Alphabetize glossary and acronyms list
\makeglossaries


% Command to conditionally typeset a bibliography.
\newcommand{\standaloneBib}{%%
  \ifthenelse{\boolean{standaloneFlag}}%%
  {\printbibliography[heading=bibintoc]
        \printglossary[type=symbols]
        \printglossary[type=acronymtype]
  \printglossary[type=main]}{}}

\DTMsavedate{presentation}{2022-11-19}

%%%%%%%%%%%%%%%%%%%
%%%%%%%%%%%%%%%% Acronym Only
\newglossaryentry{ndfeb}
{
    type=\acronymtype,
    name={NdFeB},
    description={Neodymium Iron Boron Rare Earth Permanent Magnet},
    first={Neodymium iron boron (NdFeB)}
}

%%%%%%%%%%%%%%%%%%%
%%%%%%%%%%%%%%%% Glossary Only
\newglossaryentry{cpp}
{%
    name={C++},
    description={C++ is a programming language that can be used as an object oriented programming language, an imperative programming language, and still provide low-level memory control. Note: All C++ code used in this work is compiled under the C++11 standard}
}
%%%%%%%%%%%%%%%%%%%
%%%%%%%%%%%%%%%% Glossary and Acronym
% Self referencing glossary entry to minimize what needs to be edited between changes
\newglossaryentry{cwvmg}
{
    name={\glsentrytext{cwvm}},
    description={\glsentrydesc{cwvm} (\glsentrytext{cwvm}) is a voltage multiplier that can be cascaded to give an output voltage of \(nV_{p-p}\) with \(n\) being the number of stages}
}
\newglossaryentry{cwvm}
{
    type=\acronymtype,
    name={CWVM},
    description={Cockroft-Walton voltage multiplier},
    first={\glsentrydesc{cwvm} (\glsentrytext{cwvm})\glsadd{cwvmg}},
    plural={\glsentrytext{cwvm}s},
    descriptionplural={\glsentrydesc{cwvm}s},
    firstplural={\glsentrydescplural{cwvm} (\glsentryplural{cwvm})},
    see=[Glossary:]{cwvmg}
}

%%% Constants
%%%%%%%%%%%%%%%%%%%%%%%%%%%%%%%%%%%%%%%%%%%%%%%%%%%%%%%%%%%%%%%%%%%%%%%%%%%%%%%%%%%%%%%%

%%% Symbols
%%%%%%%%%%%%%%%%%%%%%%%%%%%%%%%%%%%%%%%%%%%%%%%%%%%%%%%%%%%%%%%%%%%%%%%%%%%%%%%%%%%%%%%%
\newglossaryentry{nfdsb}
{
    type=symbols,
    name={\ensuremath{NF_{dsb}}},
    description={Noise Figure (double side band)}
}


\begin{document}{}
    \section{Introduction}
        A running theme through many of these presentation has been the idea of reuseable code.
        The idea being that once you code something up in a way that you like, say for example your preamble, you can reuse it the next time you're writing a report.
        But this is code, so we want to simplify this as much as possible.
        In fact, we don't ever want to copy and paste, we want it to be set up so that if something goes wrong and we fix some issue that fix propagates out everywhere.
        So how do we do that?
        \LaTeX~has a variety of commands and packages that facilitate this sort of thing at different levels as we explore these, keep in mind a couple of different ideas:
        \begin{itemize}
            \item Reusing material from your report in your presentation
            \item Modularizing documents; especially big documents
            \item Reusing general code
        \end{itemize}
    \section{\LaTeX~Macro Expansion}
    \section{A (very) Brief Pathing Primer}
        When setting up a modular document, often times its desireable to have it use relative paths \mintinline{bash}{../../your-file.tex} instead of absolute ones \mintinline{bash}{/home/your-user-name/your/path/to/file.tex}.
        Figure \ref{fig:directoryStructure} shows a sample of the directory for this report.
        \ExecuteMetaData[\importPath/assets.tex]{directoryStructure}
    \section{Handling Paths Yourself}
        My preferred solution to the pathing issue is to define some variables and simply define the relative distance to the top level in each file independently.
        This way, no matter which file or subfile I'm using, the correct path is being followed, and only one variable needs to be set for a file.
        \begin{listing}[H]
            \begin{centering}
                \begin{minted}{latex}
                    \providecommand{\toplevel}{../..}
                    \providecommand{\importPath}{\toplevel/Shared/Imports}
                    \providecommand{\assetPath}{\toplevel/Assets}
                    \providecommand{\sharedPath}{\toplevel/Shared}

                    \documentclass[hidelinks, float=false, crop=false]{standalone}

                    \usepackage[mode=build, subpreambles=false]{standalone}
\usepackage{import} % needed for importing with path consideration
\usepackage{geometry} % useful for defining page geometries
\usepackage{hyperref} % used for creating hyperlinks in documents. Both to the web and within the document itself
\makeatletter
\@ifclassloaded{beamer}{
  \typeout{
    -------------------------------------------------
    -------------------------------------------------
    this file is a beamer file, skipping AMS packages
    -------------------------------------------------
    -------------------------------------------------
  }
}{
  \typeout{
    ----------------------------------------------------
    ----------------------------------------------------
    this file is not a beamer file, loading AMS packages
    ----------------------------------------------------
    ----------------------------------------------------
  }
  \usepackage[tbtags]{amsmath} % for typesetting math (American Mathematical Society)
  \usepackage{amsfonts} % fonts and mathematical symbols
  \usepackage{amssymb} % more mathematical symbols
}
\makeatother
% Not needed with lualatex
% \usepackage[utf8]{inputenc} % how to treat the written file (as utf8)
% \usepackage{morewrites} % important because with the glossaries, and beamers own files it is writing out, latex can use up its own hardcoded limit
\usepackage[T1]{fontenc} % the encoding for the output file T1 is the most common, includes accents and many other commonly needed/used characters
\usepackage{hologo}
\usepackage[style=ieee,backend=biber]{biblatex} % for handling bibliographies
\usepackage{float} % added control over
\usepackage{graphicx} % tools for inclusion of graphics
\usepackage{booktabs} % adding commands to improve the look of tables
\usepackage{csvsimple} % simplify table creation by importing .csv files directly
\usepackage{siunitx} % consistent notation and correct formatting of units
\DeclareSIUnit{\torr}{Torr} % Custom unit definition
\usepackage{minted} % inclusion of code blocks with syntax highlighting and
\setminted{linenos, autogobble, fontsize=\footnotesize, breaklines=true} % set global options for minted environments
\usepackage{chemformula} % for writing chemical formulae
\usepackage[useregional]{datetime2}
\usepackage{catchfilebetweentags} % taking elements from between tags

\usepackage[debug, toc, section=section, acronym, symbols]{glossaries} % Glossaries package
\usepackage{tikz}
    \usetikzlibrary{math, arrows, circuits.ee.IEC, positioning, shapes.arrows, shapes.geometric, automata}
\usepackage{pgfplots}
    \pgfplotsset{compat=newest, compat/show suggested version=false}
    \usepgfplotslibrary{groupplots}
\usepackage[siunitx,american voltages, american currents, RPvoltages]{circuitikz}

% A command for less cramped nested fractions
\newcommand\ddfrac[2]{\frac{\displaystyle #1}{\displaystyle #2}}

% Front matter, main matter, and back matter definitions
\def\frontmatter{%
 \pagenumbering{roman}
 \setcounter{page}{1}
 \renewcommand{\thesection}{\roman{section}}
}%
\def\mainmatter{%
 \pagenumbering{arabic}
 \setcounter{page}{1}
 \setcounter{section}{0}
 \renewcommand{\thesection}{\arabic{section}}
}%
\def\backmatter{%
 \setcounter{section}{0}
 \renewcommand{\thesection}{\alph{section}}
}%

%Get rounded wire jumps
\tikzset{
    declare function={% in case of CVS which switches the arguments of atan2
        atan3(\a,\b)=ifthenelse(atan2(0,1)==90, atan2(\a,\b), atan2(\b,\a));},
        kinky cross radius/.initial=+.125cm,
        @kinky cross/.initial=+, kinky crosses/.is choice,
        kinky crosses/left/.style={@kinky cross=-},kinky crosses/right/.style={@kinky cross=+},
        kinky cross/.style args={(#1)--(#2)}{
        to path={
          let \p{@kc@}=($(\tikztotarget)-(\tikztostart)$),
              \n{@kc@}={atan3(\p{@kc@})+180} in
          -- ($(intersection of \tikztostart--{\tikztotarget} and #1--#2)!%
                 \pgfkeysvalueof{/tikz/kinky cross radius}!(\tikztostart)$)
          arc [ radius     =\pgfkeysvalueof{/tikz/kinky cross radius},
                start angle=\n{@kc@},
                delta angle=\pgfkeysvalueof{/tikz/@kinky cross}180 ]
          -- (\tikztotarget)}}
      }

% SI unit pkg conf
\sisetup{
    load-configurations = abbreviations,
    detect-family = true,
    per-mode = reciprocal
}%
\DeclareSIUnit{\torr}{Torr}

% Bibliography .bib file locations should maybe be local?
\addbibresource[location=local]{./accelerators.bib}

% For using the standalone package
\newboolean{standaloneFlag}
\setboolean{standaloneFlag}{true}
% Add Constants list using glossary
\newglossary[cgls]{constants}{cstog}{cstig}{Constants}

% Alphabetize glossary and acronyms list
\makeglossaries


% Command to conditionally typeset a bibliography.
\newcommand{\standaloneBib}{%%
  \ifthenelse{\boolean{standaloneFlag}}%%
  {\printbibliography[heading=bibintoc]
        \printglossary[type=symbols]
        \printglossary[type=acronymtype]
  \printglossary[type=main]}{}}

\DTMsavedate{presentation}{2022-11-19}

                    %%%%%%%%%%%%%%%%%%%
%%%%%%%%%%%%%%%% Acronym Only
\newglossaryentry{ndfeb}
{
    type=\acronymtype,
    name={NdFeB},
    description={Neodymium Iron Boron Rare Earth Permanent Magnet},
    first={Neodymium iron boron (NdFeB)}
}

%%%%%%%%%%%%%%%%%%%
%%%%%%%%%%%%%%%% Glossary Only
\newglossaryentry{cpp}
{%
    name={C++},
    description={C++ is a programming language that can be used as an object oriented programming language, an imperative programming language, and still provide low-level memory control. Note: All C++ code used in this work is compiled under the C++11 standard}
}
%%%%%%%%%%%%%%%%%%%
%%%%%%%%%%%%%%%% Glossary and Acronym
% Self referencing glossary entry to minimize what needs to be edited between changes
\newglossaryentry{cwvmg}
{
    name={\glsentrytext{cwvm}},
    description={\glsentrydesc{cwvm} (\glsentrytext{cwvm}) is a voltage multiplier that can be cascaded to give an output voltage of \(nV_{p-p}\) with \(n\) being the number of stages}
}
\newglossaryentry{cwvm}
{
    type=\acronymtype,
    name={CWVM},
    description={Cockroft-Walton voltage multiplier},
    first={\glsentrydesc{cwvm} (\glsentrytext{cwvm})\glsadd{cwvmg}},
    plural={\glsentrytext{cwvm}s},
    descriptionplural={\glsentrydesc{cwvm}s},
    firstplural={\glsentrydescplural{cwvm} (\glsentryplural{cwvm})},
    see=[Glossary:]{cwvmg}
}

                    %%% Constants
%%%%%%%%%%%%%%%%%%%%%%%%%%%%%%%%%%%%%%%%%%%%%%%%%%%%%%%%%%%%%%%%%%%%%%%%%%%%%%%%%%%%%%%%

%%% Symbols
%%%%%%%%%%%%%%%%%%%%%%%%%%%%%%%%%%%%%%%%%%%%%%%%%%%%%%%%%%%%%%%%%%%%%%%%%%%%%%%%%%%%%%%%
\newglossaryentry{nfdsb}
{
    type=symbols,
    name={\ensuremath{NF_{dsb}}},
    description={Noise Figure (double side band)}
}

                \end{minted}
                \caption{Manually solving the pathing issue}
                \label{lst:bibliography}
            \end{centering}
        \end{listing}

    \section{input Command}
        \mintinline{latex}{\input{you-file.tex}} is the basic form of importing. Requires no external packages, but also doesn't do anything fancy.
        Grabs the file and drops it in-place.
        \begin{itemize}
            \item Reusing your preambles (without copying and pasting)
            \item Reusing glossaries
        \end{itemize}
    \section{import Package}
        Can aid with pathing, because it separated out the file from its path and so can find files imported from subimports following their source directory.
    \section{standalone Package}
        The major limitation of the import and input commands is that if they're being use with files you'd like to compile separately (for testing, or for any other reason) having multiple preambles will create break your files.
        The standalone package is a sophisticated tool that solves this by (optionally) ignoring the preambles of any files you're pulling in (or by combining preambles).
        This brings us closer to our modularized document because it means TiKz figures can be written separately, compiled, tested and all the rest without having to compile your entire document every time.
        As an added bonus, it introduces the \mintinline{latex}{\includestandalone[\width=0.5\textwidth]{your-tikz-image}} which as shown, allows you to scale TiKz figures as though the were regular images.
        Very handy.
        It also allows you to break your report into multiple chapters.
        \begin{itemize}
            \item Breaking your document into modular parts
            \item TikZ Images
        \end{itemize}
        \subsection{A Useful Additional Command}
            When working with a modularized document, it is often desireable to compile it with glossaries or bibliographies added at the end.
            Of course you can't just add the commands \mintinline{latex}{\printbibliography} because then you would have duplicate bibliographies printed in the super file.
            My preferred approach to solving this is through the use of boolean flags.
            Listing \ref{lst:conditionalBibsForSubFiles}, gives the code to be added to the preamble that creates a command to conditionally input a bibliography if the boolean flag \mintinline{latex}{standaloneFlag} is set to \mintinline{latex}{true}.
            \begin{listing}[H]
                \begin{centering}
                    \begin{minted}{latex}
                        % For using the standalone package
                        \newboolean{standaloneFlag}
                        \setboolean{standaloneFlag}{true}
                        % Command to conditionally typeset a bibliography.
                        \newcommand{\standaloneBib}{%%
                        \ifthenelse{\boolean{standaloneFlag}}%%
                        {\printbibliography[heading=bibintoc]
                                \printglossary[type=symbols]
                                \printglossary[type=acronymtype]
                        \printglossary[type=main]}{}}
                    \end{minted}
                    \caption{Conditionally typeset bibliography and glossaries}
                    \label{lst:conditionalBibsForSubFiles}
                \end{centering}
            \end{listing}
            In the subfiles, now you can issue the command \mintinline{latex}{\standaloneBib} to have the bibliography and glossaries printed if you in the subfile.
            In the super file then, issue the command \mintinline{latex}{\setboolean{standaloneFlag}{false}} after loading the preamble, to have all instances of \mintinline{latex}{standaloneBib} blocked, and then use the normal commands to have the bibliography and glossaries typset where you want them.


    \section{catchfilebetweentags Package}
        \subsection{catchfilebetweentags}
            This handy package allows you to store a bunch of different things (say for example, equations) and pull them in to your document. This is really nice because it makes your document's code more easily read. It also makes it that you can reuse the equations in another place, say for example an associated presentations.

            You can use it for anything from equations,
            \ExecuteMetaData[\importPath/equations.tex]{nucEnerReleased}
            \ExecuteMetaData[\importPath/equations.tex]{pairEnParam}
            \ExecuteMetaData[\importPath/equations.tex]{nucBinEner}
            to tables,
            \ExecuteMetaData[\importPath/assets.tex]{targsAndVals}
            to figures.
            \ExecuteMetaData[\importPath/assets.tex]{tikzCircuitDCDC}
            \ExecuteMetaData[\importPath/assets.tex]{pnjunction}
            \ExecuteMetaData[\importPath/assets.tex]{mosfettikz}
            \ExecuteMetaData[\importPath/assets.tex]{tikzLorenz}
            \ExecuteMetaData[\importPath/equations.tex]{lorenzAttr}
            Handy!

    \clearpage
    \standaloneBib
\end{document}
