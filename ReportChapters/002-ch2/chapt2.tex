% arara: lualatex: { shell: true, interaction: nonstopmode }
% arara: makeglossaries
% arara: biber
% arara: lualatex: { shell: true, interaction: nonstopmode }
% arara: lualatex: { shell: true, interaction: nonstopmode }

\providecommand{\toplevel}{../..}
\providecommand{\importPath}{\toplevel/Shared/Imports}
\providecommand{\assetPath}{\toplevel/Assets}
\providecommand{\sharedPath}{\toplevel/Shared}

\documentclass[hidelinks]{article}

%%%%%%%%%%%%%%%%%%%%%%%%%%%%%%%%%%%%%%%%%%%%%%%%%%%%%%%%%%%%%%%%%%%%%%%%%%%%%%%%%%%%%%%%%%%%%%%%%%%%%%%%%%%%%%%%%%%%%%%%%%%%%%%%%%%%%%%%%%%%%%%%%%%%%%%%%%%%%%%%%
%%%%%%%%%% LOAD PACKAGES %%%%%%%%%%%%%%%%%%%%%%%%%%%%%%%%%%%%%%%%%%%%%%%%%%%%%%%%%%%%%%%%%%%%%%%%%%%%%%%%%%%%%%%%%%%%%%%%%%%%%%%%%%%%%%%%%%%%%%%%%%%%%%%%%%%%%%%%
%%%%%%%%%%%%%%%%%%%%%%%%%%%%%%%%%%%%%%%%%%%%%%%%%%%%%%%%%%%%%%%%%%%%%%%%%%%%%%%%%%%%%%%%%%%%%%%%%%%%%%%%%%%%%%%%%%%%%%%%%%%%%%%%%%%%%%%%%%%%%%%%%%%%%%%%%%%%%%%%%
\usepackage{geometry} % useful for defining page geometries
\usepackage{hyperref} % used for creating hyperlinks in documents. Both to the web and within the document itself
\usepackage[mode=build, subpreambles=false]{standalone} % for using the standalone package for import
\makeatletter
\@ifclassloaded{beamer}{
  \typeout{
    -------------------------------------------------
    -------------------------------------------------
    this file is a beamer file, skipping AMS packages
    -------------------------------------------------
    -------------------------------------------------
  }
}{
  \typeout{
    ----------------------------------------------------
    ----------------------------------------------------
    this file is not a beamer file, loading AMS packages
    ----------------------------------------------------
    ----------------------------------------------------
  }
  \usepackage[tbtags]{amsmath} % for typesetting math (American Mathematical Society)
  \usepackage{amsfonts} % fonts and mathematical symbols
  \usepackage{amssymb} % more mathematical symbols
}
\makeatother
% These packages aren't needed when compiling under LuaLaTeX
% \usepackage[utf8]{inputenc} % how to treat the written file (as utf8)
% \usepackage{morewrites} % important because with the glossaries, latex can use up its own hardcoded limit of files it can write out
% \usepackage[T1]{fontenc} % the encoding for the output file T1 is the most common, includes accents and many other commonly needed/used characters
%%%%%%%%%%%%%%%%%%%%%%%%%%%%%%%%%%%%%%%%%%%%%%%%%%%%%%%%%%%%%%%%%%%%%%%%%%%%%%%%%%%%%%%%%%%%%%%%%%%%%%%%%%%%%%%%%%%%%%%%%%%%%%%%%%%%%%%%%%%%%%%%%%%%%%%%%%%%%%%%%
\usepackage[style=ieee,backend=biber]{biblatex} % for handling bibliographies
\usepackage{float} % added control over
\usepackage{graphicx} % tools for inclusion of graphics
\usepackage{booktabs} % adding commands to improve the look of tables
\usepackage{csvsimple} % simplify table creation by importing .csv files directly
\usepackage{siunitx} % consistent notation and correct formatting of units
\usepackage{minted} % inclusion of code blocks with syntax highlighting and
\usepackage{xcolor} % to access the named colour LightGray
\usepackage{chemformula} % for writing chemical formulae
\usepackage[useregional]{datetime2} % facilitate date and time typesetting
\usepackage{catchfilebetweentags} % taking elements from between tags
\usepackage[debug, toc, section=section, acronym, symbols]{glossaries} % Glossaries package
\usepackage{tikz} % to produce tikz vector graphics images in your document
\usepackage{pgfplots} % to produce tikz/pgf vector graphic plots within your document
\usepackage[siunitx,american voltages, american currents, RPvoltages]{circuitikz} % circuits with tikz
\usepackage{hologo}
\usepackage{luacode}

%%%%%%%%%%%%%%%%%%%%%%%%%%%%%%%%%%%%%%%%%%%%%%%%%%%%%%%%%%%%%%%%%%%%%%%%%%%%%%%%%%%%%%%%%%%%%%%%%%%%%%%%%%%%%%%%%%%%%%%%%%%%%%%%%%%%%%%%%%%%%%%%%%%%%%%%%%%%%%%%%
%%%%%%%%%% PACKAGE SETUP %%%%%%%%%%%%%%%%%%%%%%%%%%%%%%%%%%%%%%%%%%%%%%%%%%%%%%%%%%%%%%%%%%%%%%%%%%%%%%%%%%%%%%%%%%%%%%%%%%%%%%%%%%%%%%%%%%%%%%%%%%%%%%%%%%%%%%%%
%%%%%%%%%%%%%%%%%%%%%%%%%%%%%%%%%%%%%%%%%%%%%%%%%%%%%%%%%%%%%%%%%%%%%%%%%%%%%%%%%%%%%%%%%%%%%%%%%%%%%%%%%%%%%%%%%%%%%%%%%%%%%%%%%%%%%%%%%%%%%%%%%%%%%%%%%%%%%%%%%
% minted
\definecolor{LightGray}{gray}{0.9}
\colorlet{PresentationBlack}{black!2}
\setminted{linenos=false, autogobble, fontsize=\tiny, breaklines=true, bgcolor=LightGray, breakbefore=\{} % global default options for minted
\setmintedinline{bgcolor={}} % global default options for inline minted

% tikz
\usetikzlibrary{math, arrows, circuits.ee.IEC, positioning, shapes.arrows, shapes.geometric, automata, fadings}

% pgfplots
\pgfplotsset{compat=newest, compat/show suggested version=false}
\usepgfplotslibrary{groupplots}

% siunitx
\sisetup{
detect-family = true,
detect-weight = true,
per-mode = reciprocal,
input-digits = { 0123456789\pi\dots },
input-comparators = { <=>\approx\ge\geq\gg\le\leq\ll\sim\gtrsim\lesssim },
table-auto-round,
table-align-comparator = true,
}%
\DeclareSIUnit{\torr}{Torr} % Custom unit definition

% Bibliography .bib file location
\addbibresource[location=local]{\bibPath/references.bib}

% Add Constants list using glossary
\newglossary[cgls]{constants}{cstog}{cstig}{Constants}

% Alphabetize glossary and acronyms list
\makeglossaries

%%%%%%%%%%%%%%%%%%%%%%%%%%%%%%%%%%%%%%%%%%%%%%%%%%%%%%%%%%%%%%%%%%%%%%%%%%%%%%%%%%%%%%%%%%%%%%%%%%%%%%%%%%%%%%%%%%%%%%%%%%%%%%%%%%%%%%%%%%%%%%%%%%%%%%%%%%%%%%%%%
%%%%%%%%%% ADDITIONAL SETUP %%%%%%%%%%%%%%%%%%%%%%%%%%%%%%%%%%%%%%%%%%%%%%%%%%%%%%%%%%%%%%%%%%%%%%%%%%%%%%%%%%%%%%%%%%%%%%%%%%%%%%%%%%%%%%%%%%%%%%%%%%%%%%%%%%%%%
%%%%%%%%%%%%%%%%%%%%%%%%%%%%%%%%%%%%%%%%%%%%%%%%%%%%%%%%%%%%%%%%%%%%%%%%%%%%%%%%%%%%%%%%%%%%%%%%%%%%%%%%%%%%%%%%%%%%%%%%%%%%%%%%%%%%%%%%%%%%%%%%%%%%%%%%%%%%%%%%%
% A command for less cramped nested fractions
\newcommand\ddfrac[2]{\frac{\displaystyle #1}{\displaystyle #2}}

% Front matter, main matter, and back matter definitions (not needed for document class book or memoire)
\def\frontmatter{%
 \pagenumbering{roman}
 \setcounter{page}{1}
 \renewcommand{\thesection}{\roman{section}}
}%
\def\mainmatter{%
 \pagenumbering{arabic}
 \setcounter{page}{1}
 \setcounter{section}{0}
 \renewcommand{\thesection}{\arabic{section}}
}%
\def\backmatter{%
 \setcounter{section}{0}
 \renewcommand{\thesection}{\alph{section}}
}%

%Get rounded wire jumps
\tikzset{
    declare function={% in case of CVS which switches the arguments of atan2
        atan3(\a,\b)=ifthenelse(atan2(0,1)==90, atan2(\a,\b), atan2(\b,\a));},
        kinky cross radius/.initial=+.125cm,
        @kinky cross/.initial=+, kinky crosses/.is choice,
        kinky crosses/left/.style={@kinky cross=-},kinky crosses/right/.style={@kinky cross=+},
        kinky cross/.style args={(#1)--(#2)}{
        to path={
          let \p{@kc@}=($(\tikztotarget)-(\tikztostart)$),
              \n{@kc@}={atan3(\p{@kc@})+180} in
          -- ($(intersection of \tikztostart--{\tikztotarget} and #1--#2)!%
                 \pgfkeysvalueof{/tikz/kinky cross radius}!(\tikztostart)$)
          arc [ radius     =\pgfkeysvalueof{/tikz/kinky cross radius},
                start angle=\n{@kc@},
                delta angle=\pgfkeysvalueof{/tikz/@kinky cross}180 ]
          -- (\tikztotarget)}},
    onslide/.code args={<#1>#2}{%
        \only<#1>{\pgfkeysalso{#2}} % \pgfkeysalso doesn't change the path
    },
    myfading/.style n args={2}{
        postaction={
            decorate,
            decoration={
                markings,
                mark=between positions 0 and \pgfdecoratedpathlength-4pt step 0.2pt with {
                    \pgfmathsetmacro\myval{
                            multiply(
                                divide(
                                    \pgfkeysvalueof{/pgf/decoration/mark info/distance from start},
                                    \pgfdecoratedpathlength
                                ),
                                100
                            )
                    };
                    \pgfsetfillcolor{#2!\myval!#1};
                    \pgfpathcircle{\pgfpointorigin}{\pgflinewidth};
                    \pgfusepath{fill};
                },
                mark=at position \pgfdecoratedpathlength with {\arrow[#2, >=latex, xshift=0.35\pgflinewidth]{>}},
            }
        }
    },
}
\usepackage[most]{tcolorbox}
\usepackage{tikzpagenodes}

\def\myblur{4}

\makeatletter
\newtcolorbox{blur}[1][]{%
  #1,
  enhanced,
  remember,
  frame hidden,
  interior hidden,
  fonttitle=\bfseries,
  coltitle=black,
  underlay={
    \begin{tcbclipframe}
      \begin{scope}[remember picture,overlay,inner sep=0pt]
        \fill[white] (current page.south west) rectangle (current page.north east);
        \foreach \x in {-10,-7.5,...,10}{
        \foreach \y in {-10,-7.5,...,10}{
          \node[opacity=0.01] at ([yshift=\y,xshift=\x]current page.center) {\includestandalone[mode=tex, width=\textwidth]{\assetPath/Images/Tikz/texfamilytree/texfamilytree}};
        }}
      \end{scope}
    \end{tcbclipframe}
   }
}
\makeatother

\def\checkmark{\tikz\fill[scale=0.4](0,.35) -- (.25,0) -- (1,.7) -- (.25,.15) -- cycle;}

% For using the standalone package and conditionally typesetting bibliography and glossaries
\newboolean{standaloneFlag}
\setboolean{standaloneFlag}{true}
% Create command to conditionally typeset a bibliography.
\newcommand{\standaloneBib}{%%
  \ifthenelse{\boolean{standaloneFlag}}{
    \clearpage
    \printbibliography[heading=bibintoc]
    \printglossary[type=symbols]
    \printglossary[type=constants]
    \printglossary[type=acronymtype]
    \printglossary[type=main]
  }{}}

% Code for Lua search and Replace

\directlua{
    path = "\luacodePath"
    demo = require(path.."/searchandreplace")
}
\newcommand\luastringsubs[2]{%
    \directlua{
        replacementsPreTable = [[#1]]
        filepath = [[#2]]
        subbedString = stringReplaceFromFile(replacementsPreTable, filepath)
        -- print(subbedString)
        -- tex.sprint(subbedString)
    }%
}

%%%%%%%%%%%%%%%%%%%
%%%%%%%%%%%%%%%% Acronym Only
\newglossaryentry{ndfeb}
{
    type=\acronymtype,
    name={NdFeB},
    description={Neodymium Iron Boron Rare Earth Permanent Magnet},
    first={Neodymium iron boron (NdFeB)}
}

%%%%%%%%%%%%%%%%%%%
%%%%%%%%%%%%%%%% Glossary Only
\newglossaryentry{cpp}
{%
    name={C++},
    description={C++ is a programming language that can be used as an object oriented programming language, an imperative programming language, and still provide low-level memory control. Note: All C++ code used in this work is compiled under the C++11 standard}
}
%%%%%%%%%%%%%%%%%%%
%%%%%%%%%%%%%%%% Glossary and Acronym
% Self referencing glossary entry to minimize what needs to be edited between changes
\newglossaryentry{cwvmg}
{
    name={\glsentrytext{cwvm}},
    description={\glsentrydesc{cwvm} (\glsentrytext{cwvm}) is a voltage multiplier that can be cascaded to give an output voltage of \(nV_{p-p}\) with \(n\) being the number of stages}
}
\newglossaryentry{cwvm}
{
    type=\acronymtype,
    name={CWVM},
    description={Cockroft-Walton voltage multiplier},
    first={\glsentrydesc{cwvm} (\glsentrytext{cwvm})\glsadd{cwvmg}},
    plural={\glsentrytext{cwvm}s},
    descriptionplural={\glsentrydesc{cwvm}s},
    firstplural={\glsentrydescplural{cwvm} (\glsentryplural{cwvm})},
    see=[Glossary:]{cwvmg}
}

%%% Constants
%%%%%%%%%%%%%%%%%%%%%%%%%%%%%%%%%%%%%%%%%%%%%%%%%%%%%%%%%%%%%%%%%%%%%%%%%%%%%%%%%%%%%%%%

%%% Symbols
%%%%%%%%%%%%%%%%%%%%%%%%%%%%%%%%%%%%%%%%%%%%%%%%%%%%%%%%%%%%%%%%%%%%%%%%%%%%%%%%%%%%%%%%
\newglossaryentry{nfdsb}
{
    type=symbols,
    name={\ensuremath{NF_{dsb}}},
    description={Noise Figure (double side band)}
}


\begin{document}
    \section{The \LaTeX~Family Tree}
        The \TeX~family tree is a large one. You will often see many of these terms thrown around and it can be difficult to make sense of it.
        The tree shown in Figure \ref{fig:texfamilytree} shows how each of the major active \TeX~siblings relate.
        \ExecuteMetaData[\importPath/assets.tex]{texfamilytree}
        \subsection{\TeX}
            \TeX~is the original typsetting tool create by Donald Knuth in the 1980s.
        \subsection{\Hologo{pdfTeX}}
            \Hologo{pdfTeX} is an extension to the original \TeX~that enables the creation of PDF files.
        \subsection{\LaTeX}
            As mention \TeX~primitive are the real way of interacting with the \TeX~engine, but they are challenging to work with.
            \LaTeX~is a collection of macros that ease the use \TeX~and facilitate writing new packages.
        \subsection{\Hologo{XeTeX}}
            \Hologo{XeTeX} is a development on \TeX~that extends support for languages and glyphs beyond those using just the roman alphabet.
        \subsection{\Hologo{LuaTeX}}
            \Hologo{LuaTeX} is a recent and ongoing development which exposes the \TeX~primitives via the small and fast scripting language Lua.
            This has many benefits but can be effectively summarized as making it easier to code in \TeX.

    \section{Processing \TeX~Files}
        \subsection{Manually Processing Files}
            Processing a \TeX~file typically just means running the right programs in a certain order.
            Because of the sizing, number, and ordering work \LaTeX~is doing during the compilation, it needs to be run at least twice.
            Typically if our document includes a bibliography (and other elements like glossaries or indices), the compilation procedure would mean executing each of the commands shown in Listing \ref{lst:compilation}.
            \begin{listing}[htbp]
                \begin{centering}
                    \begin{minted}{bash}
                        pdflatex --shell-escape --interaction=nonstopmode report
                        biber report
                        makeglossaries report
                        pdflatex --shell-escape --interaction=nonstopmode report
                        pdflatex --shell-escape --interaction=nonstopmode report
                    \end{minted}
                    \caption{Shell commands needed to compile a report directly}
                    \label{lst:compilation}
                \end{centering}
            \end{listing}
            A neat detail of these commands is that they can accept any \LaTeX~commands.
            This gives the possibility of changing variables in a document during the execution (see Listing \ref{lst:compilationwithcommands}).
            \begin{listing}[htbp]
                \begin{centering}
                    \begin{minted}{bash}
                        lualatex --shell-escape --interaction=nonstopmode "\\providecommand{\\iswhichmode}{draft}\\input{report}"
                        biber report
                        makeglossaries report
                        lualatex --shell-escape --interaction=nonstopmode "\\providecommand{\\iswhichmode}{draft}\\input{report}"
                        lualatex --shell-escape --interaction=nonstopmode "\\providecommand{\\iswhichmode}{final}\\input{report}"
                    \end{minted}
                    \caption{Shell commands compiling a document with addition commands provided at compile time}
                    \label{lst:compilationwithcommands}
                \end{centering}
            \end{listing}

        \subsection{latexmk}
            latexmk is a tool that tries to automate the latex compilation process by reading in the log files and figuring out what additional programs need to be run, and when to rerun the compilation.
            It is a very effective tool and in most cases the defacto \LaTeX~compiler.
            \begin{listing}[htbp]
                \begin{centering}
                    \begin{minted}{bash}
                        latexmk -pdf report.tex
                    \end{minted}
                    \caption{Shell command for compiling with latexmk}
                    \label{lst:latexmkcommands}
                \end{centering}
            \end{listing}

        \subsection{arara}
            Sometimes however, latexmk doesn't know of some additional tool or new intermediate program that needs to be run, and this can create issues.
            It is also at times trying to be too smart, and ends up creating issues.
            For these reasons my go-to for a number of years has been ARARA, a compilation tool that allows you to define the compilation process at the beginning of your document.
            Listing \ref{lst:araradirectives} shows the directive syntax for arara, with Listing \ref{lst:araracompilation} showing the actual command.
            \begin{listing}[H]
                \begin{centering}
                    \begin{minted}{latex}
                        % arara: lualatex: { shell: true, interaction: nonstopmode }
                        % arara: makeglossaries
                        % arara: biber
                        % arara: lualatex: { shell: true, interaction: nonstopmode }
                        % arara: lualatex: { synctex: true, shell: true, interaction: nonstopmode }
                    \end{minted}
                    \caption{Shell command for compiling with Arara}
                    \label{lst:araradirectives}
                \end{centering}
            \end{listing}
            \begin{listing}[H]
                \begin{centering}
                    \begin{minted}{bash}
                        arara -v report.tex
                    \end{minted}
                    \caption{Shell command for compiling with Arara}
                    \label{lst:araracompilation}
                \end{centering}
            \end{listing}

    \section{\TeX~Distributions}
        \TeX~and \LaTeX~is a very large collection of programs and files, and installing them each individually would be a huge hassle.
        Thankfully the programs and packages are available bundled together as a single distribution.
        The distribution of choice where possible is TeXLive.
        On Linux systems it is always available in the systems package manager.
        For Mac and Windows systems, there are a few different options because sometimes it can be challenging to get TeXLive to operate (though I have no personal experience with these).
        \subsection{Mac}
            \begin{itemize}
                \item TeXLive
                \item MacTeX
            \end{itemize}
        \subsection{Windows}
            \begin{itemize}
                \item TeXLive
                \item MiKTeX
                \item ProTeXt
            \end{itemize}

    \clearpage
    \standaloneBib
\end{document}
