% arara: lualatex: { shell: true, interaction: nonstopmode }
% arara: makeglossaries
% arara: biber
% arara: lualatex: { shell: true, interaction: nonstopmode }
% arara: lualatex: { shell: true, interaction: nonstopmode }

\providecommand{\toplevel}{../..}
\providecommand{\importPath}{\toplevel/Shared/Imports}
\providecommand{\assetPath}{\toplevel/Assets}
\providecommand{\sharedPath}{\toplevel/Shared}

\documentclass[hidelinks, float=false, crop=false]{standalone}

\usepackage[mode=build, subpreambles=false]{standalone}
\usepackage{import} % needed for importing with path consideration
\usepackage{geometry} % useful for defining page geometries
\usepackage{hyperref} % used for creating hyperlinks in documents. Both to the web and within the document itself
\makeatletter
\@ifclassloaded{beamer}{
  \typeout{
    -------------------------------------------------
    -------------------------------------------------
    this file is a beamer file, skipping AMS packages
    -------------------------------------------------
    -------------------------------------------------
  }
}{
  \typeout{
    ----------------------------------------------------
    ----------------------------------------------------
    this file is not a beamer file, loading AMS packages
    ----------------------------------------------------
    ----------------------------------------------------
  }
  \usepackage[tbtags]{amsmath} % for typesetting math (American Mathematical Society)
  \usepackage{amsfonts} % fonts and mathematical symbols
  \usepackage{amssymb} % more mathematical symbols
}
\makeatother
% Not needed with lualatex
% \usepackage[utf8]{inputenc} % how to treat the written file (as utf8)
% \usepackage{morewrites} % important because with the glossaries, and beamers own files it is writing out, latex can use up its own hardcoded limit
\usepackage[T1]{fontenc} % the encoding for the output file T1 is the most common, includes accents and many other commonly needed/used characters
\usepackage{hologo}
\usepackage[style=ieee,backend=biber]{biblatex} % for handling bibliographies
\usepackage{float} % added control over
\usepackage{graphicx} % tools for inclusion of graphics
\usepackage{booktabs} % adding commands to improve the look of tables
\usepackage{csvsimple} % simplify table creation by importing .csv files directly
\usepackage{siunitx} % consistent notation and correct formatting of units
\DeclareSIUnit{\torr}{Torr} % Custom unit definition
\usepackage{minted} % inclusion of code blocks with syntax highlighting and
\setminted{linenos, autogobble, fontsize=\footnotesize, breaklines=true} % set global options for minted environments
\usepackage{chemformula} % for writing chemical formulae
\usepackage[useregional]{datetime2}
\usepackage{catchfilebetweentags} % taking elements from between tags

\usepackage[debug, toc, section=section, acronym, symbols]{glossaries} % Glossaries package
\usepackage{tikz}
    \usetikzlibrary{math, arrows, circuits.ee.IEC, positioning, shapes.arrows, shapes.geometric, automata}
\usepackage{pgfplots}
    \pgfplotsset{compat=newest, compat/show suggested version=false}
    \usepgfplotslibrary{groupplots}
\usepackage[siunitx,american voltages, american currents, RPvoltages]{circuitikz}

% A command for less cramped nested fractions
\newcommand\ddfrac[2]{\frac{\displaystyle #1}{\displaystyle #2}}

% Front matter, main matter, and back matter definitions
\def\frontmatter{%
 \pagenumbering{roman}
 \setcounter{page}{1}
 \renewcommand{\thesection}{\roman{section}}
}%
\def\mainmatter{%
 \pagenumbering{arabic}
 \setcounter{page}{1}
 \setcounter{section}{0}
 \renewcommand{\thesection}{\arabic{section}}
}%
\def\backmatter{%
 \setcounter{section}{0}
 \renewcommand{\thesection}{\alph{section}}
}%

%Get rounded wire jumps
\tikzset{
    declare function={% in case of CVS which switches the arguments of atan2
        atan3(\a,\b)=ifthenelse(atan2(0,1)==90, atan2(\a,\b), atan2(\b,\a));},
        kinky cross radius/.initial=+.125cm,
        @kinky cross/.initial=+, kinky crosses/.is choice,
        kinky crosses/left/.style={@kinky cross=-},kinky crosses/right/.style={@kinky cross=+},
        kinky cross/.style args={(#1)--(#2)}{
        to path={
          let \p{@kc@}=($(\tikztotarget)-(\tikztostart)$),
              \n{@kc@}={atan3(\p{@kc@})+180} in
          -- ($(intersection of \tikztostart--{\tikztotarget} and #1--#2)!%
                 \pgfkeysvalueof{/tikz/kinky cross radius}!(\tikztostart)$)
          arc [ radius     =\pgfkeysvalueof{/tikz/kinky cross radius},
                start angle=\n{@kc@},
                delta angle=\pgfkeysvalueof{/tikz/@kinky cross}180 ]
          -- (\tikztotarget)}}
      }

% SI unit pkg conf
\sisetup{
    load-configurations = abbreviations,
    detect-family = true,
    per-mode = reciprocal
}%
\DeclareSIUnit{\torr}{Torr}

% Bibliography .bib file locations should maybe be local?
\addbibresource[location=local]{./accelerators.bib}

% For using the standalone package
\newboolean{standaloneFlag}
\setboolean{standaloneFlag}{true}
% Add Constants list using glossary
\newglossary[cgls]{constants}{cstog}{cstig}{Constants}

% Alphabetize glossary and acronyms list
\makeglossaries


% Command to conditionally typeset a bibliography.
\newcommand{\standaloneBib}{%%
  \ifthenelse{\boolean{standaloneFlag}}%%
  {\printbibliography[heading=bibintoc]
        \printglossary[type=symbols]
        \printglossary[type=acronymtype]
  \printglossary[type=main]}{}}

\DTMsavedate{presentation}{2022-11-19}

%%%%%%%%%%%%%%%%%%%
%%%%%%%%%%%%%%%% Acronym Only
\newglossaryentry{ndfeb}
{
    type=\acronymtype,
    name={NdFeB},
    description={Neodymium Iron Boron Rare Earth Permanent Magnet},
    first={Neodymium iron boron (NdFeB)}
}

%%%%%%%%%%%%%%%%%%%
%%%%%%%%%%%%%%%% Glossary Only
\newglossaryentry{cpp}
{%
    name={C++},
    description={C++ is a programming language that can be used as an object oriented programming language, an imperative programming language, and still provide low-level memory control. Note: All C++ code used in this work is compiled under the C++11 standard}
}
%%%%%%%%%%%%%%%%%%%
%%%%%%%%%%%%%%%% Glossary and Acronym
% Self referencing glossary entry to minimize what needs to be edited between changes
\newglossaryentry{cwvmg}
{
    name={\glsentrytext{cwvm}},
    description={\glsentrydesc{cwvm} (\glsentrytext{cwvm}) is a voltage multiplier that can be cascaded to give an output voltage of \(nV_{p-p}\) with \(n\) being the number of stages}
}
\newglossaryentry{cwvm}
{
    type=\acronymtype,
    name={CWVM},
    description={Cockroft-Walton voltage multiplier},
    first={\glsentrydesc{cwvm} (\glsentrytext{cwvm})\glsadd{cwvmg}},
    plural={\glsentrytext{cwvm}s},
    descriptionplural={\glsentrydesc{cwvm}s},
    firstplural={\glsentrydescplural{cwvm} (\glsentryplural{cwvm})},
    see=[Glossary:]{cwvmg}
}

%%% Constants
%%%%%%%%%%%%%%%%%%%%%%%%%%%%%%%%%%%%%%%%%%%%%%%%%%%%%%%%%%%%%%%%%%%%%%%%%%%%%%%%%%%%%%%%

%%% Symbols
%%%%%%%%%%%%%%%%%%%%%%%%%%%%%%%%%%%%%%%%%%%%%%%%%%%%%%%%%%%%%%%%%%%%%%%%%%%%%%%%%%%%%%%%
\newglossaryentry{nfdsb}
{
    type=symbols,
    name={\ensuremath{NF_{dsb}}},
    description={Noise Figure (double side band)}
}


\begin{document}
    \section{The \LaTeX~Family Tree}
        \ExecuteMetaData[\importPath/assets.tex]{texfamilytree}
        \subsection{\TeX}
        \subsection{\Hologo{pdfTeX}}
        \subsection{\LaTeX}
        \subsection{\Hologo{LuaTeX}}
        \subsection{\Hologo{XeTeX}}

    \section{Processing \TeX~Files}
        \subsection{Manually Processing Files}
            Processing a \TeX~file typically just means running the right programs in a certain order.
            Because of the sizing, number, and ordering work \LaTeX~is doing during the compilation, it needs to be run at least twice.
            Typically if our document includes a bibliography (and other elements like glossaries or indices), the compilation procedure would mean executing each of the commands shown in Listing \ref{lst:compilation}.
            \begin{listing}[htbp]
                \begin{centering}
                    \begin{minted}{bash}
                        pdflatex --shell-escape --interaction=nonstopmode report
                        biber report
                        makeglossaries report
                        pdflatex --shell-escape --interaction=nonstopmode report
                        pdflatex --shell-escape --interaction=nonstopmode report
                    \end{minted}
                    \caption{Shell commands needed to compile a report directly}
                    \label{lst:compilation}
                \end{centering}
            \end{listing}
            A neat detail of these commands is that they can accept any \LaTeX~commands.
            This gives the possibility of changing variables in a document during the execution (see Listing \ref{lst:compilationwithcommands}).
            \begin{listing}[htbp]
                \begin{centering}
                    \begin{minted}{bash}
                        lualatex --shell-escape --interaction=nonstopmode "\\providecommand{\\iswhichmode}{draft}\\input{report}"
                        biber report
                        makeglossaries report
                        lualatex --shell-escape --interaction=nonstopmode "\\providecommand{\\iswhichmode}{draft}\\input{report}"
                        lualatex --shell-escape --interaction=nonstopmode "\\providecommand{\\iswhichmode}{final}\\input{report}"
                    \end{minted}
                    \caption{Shell commands compiling a document with addition commands provided at compile time}
                    \label{lst:compilationwithcommands}
                \end{centering}
            \end{listing}

        \subsection{latexmk}
            latexmk is a tool that tries to automate the latex compilation process by reading in the log files and figuring out what additional programs need to be run, and when to rerun the compilation.
            It is a very effective tool and in most cases the defacto \LaTeX~compiler.
            \begin{listing}[htbp]
                \begin{centering}
                    \begin{minted}{bash}
                        latexmk -pdf report.tex
                    \end{minted}
                    \caption{Shell command for compiling with latexmk}
                    \label{lst:latexmkcommands}
                \end{centering}
            \end{listing}

        \subsection{arara}
            Sometimes however, latexmk doesn't know of some additional tool or new intermediate program that needs to be run, and this can create issues.
            It is also at times trying to be too smart, and ends up creating issues.
            For these reasons my go-to for a number of years has been ARARA, a compilation tool that allows you to define the compilation process at the beginning of your document.
            Listing \ref{lst:araradirectives} shows the directive syntax for arara, with Listing \ref{lst:araracompilation} showing the actual command.
            \begin{listing}[H]
                \begin{centering}
                    \begin{minted}{latex}
                        % arara: lualatex: { shell: true, interaction: nonstopmode }
                        % arara: makeglossaries
                        % arara: biber
                        % arara: lualatex: { shell: true, interaction: nonstopmode }
                        % arara: lualatex: { synctex: true, shell: true, interaction: nonstopmode }
                    \end{minted}
                    \caption{Shell command for compiling with Arara}
                    \label{lst:araradirectives}
                \end{centering}
            \end{listing}
            \begin{listing}[H]
                \begin{centering}
                    \begin{minted}{bash}
                        arara -v report.tex
                    \end{minted}
                    \caption{Shell command for compiling with Arara}
                    \label{lst:araracompilation}
                \end{centering}
            \end{listing}

    \section{\TeX~Distributions}
        \TeX~and \LaTeX~is a very large collection of programs and files, and installing them each individually would be a huge hassle.
        Thankfully the programs and packages are available bundled together as a single distribution.
        The distribution of choice where possible is TeXLive.
        On Linux systems it is always available in the systems package manager.
        For Mac and Windows systems, there are a few different options because sometimes it can be challenging to get TeXLive to operate (though I have no personal experience with these).
        \subsection{Mac}
            \begin{itemize}
                \item TeXLive
                \item MacTeX
            \end{itemize}
        \subsection{Windows}
            \begin{itemize}
                \item TeXLive
                \item MiKTeX
                \item ProTeXt
            \end{itemize}

    \clearpage
    \standaloneBib
\end{document}
